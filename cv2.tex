\documentstyle[12pt]{article}
\setlength{\textheight}{9.80in}
\setlength{\textwidth}{6.40in}
\setlength{\oddsidemargin}{0.0mm}
\setlength{\evensidemargin}{1.0mm}
\setlength{\topmargin}{-0.6in}
\setlength{\parindent}{0.0in}
\setlength{\parskip}{1.5ex}
\newtheorem{defn}{Definition}
\renewcommand{\baselinestretch}{1}

\begin{document}

\thispagestyle{empty}

\title{Curriculum Vitae} 
\maketitle


\center{
\begin{tabular}{|l|l|} \hline
{\bf first name } & {\bf surname  } \\
  Christopher  &   Godsalve \\ 
{\bf Date of Birth} &  \\ 6th August 1959 &  \\
{\bf Nationality} &  {\bf Marital Status} \\
British &  Married \\
{\bf Home Address} &  \\ 
42 Swainstone Road   &     \\
Reading   &  \\ 
 RG2 ODX & \\
 {\bf Tel. No} & (0118) 9864 389 \\
\hline                    
\end{tabular}
}
\begin{flushleft}
\section{Education}

1970-1975   Hugh Farringdon (Reading) secondary modern \\
Oct 1981 - June 1983 Reading Technical College (Evening Classes) \newline
A-Level Mathematics (C) 
A Level Physics (B) \newline
Oct 1983 - June 1986  Heriot Watt University \newline
Bsc Physics (Upper Second) \newline
Oct 1986 - June 1989 Heriot Watt University  \newline
Ph.D. by research in Physics 

\section{Introduction}

When I was doing labouring jobs in 1981 I decided to enrol at evening classes to do A level
mathematics and physics. This was out of pure interest, however it led me to go to university
where I did a PhD by research. A few details are outlined below, including the general skills
I learnt on the way. This led to a research career of some 13 years. During this period
I married, had a daughter, and moved to Reading.

Ten years ago, I started suffering from severe IBD, and gradually my research began to
suffer. As a result, I have not worked since 2002. However, over the last couple of years
my consultant has finally managed to find a medication regime that works, and I wish to return to
work. However, with the way that academic research is funded, it is impossible to return
to academic research, and so I need  a new career.

Oddly enough, I still recall someone asking me, what can you possibly do with A-Level physics,
and though I hadn't thought about it, being an optician immediately sprung to mind. I was reminded 
of this when I came across your advertisement in the "Chronicle Extra". Although I would appear to
be very over-qualified I am certainly interested in the position of Assistant Dispensing Optician.


\section{Employment}

1975-1983 I started an apprenticeship in engineering, however the environment gave me severe 
eczema and I had to leave. After this I carried out 
diverse unskilled and semi skilled work. \newline
Oct 1989 -  March 1991   Computer Programmer \newline
Department of Mathematics, Heriot Watt University \newline
March 1991  -  May 1992  
Research Assistant \newline
Department of Physics, Heriot Watt University \newline
Research Fellow RA(1B/1A) \newline
NUTIS/ESSC (University of Reading),    April 1992 - October 2002 \newline


\end{flushleft}



\begin{flushleft}
\section{Research Experience}

My Ph.D. research was related to Heriot-Watt university's optical computing project
in which nonlinear devices switched laser beams. My research was theoretical and consisted of mathematical
modelling and numerical computer programming in FORTRAN 77.
My most enjoyable times involved collaboration with 
with different experimental groups.  These were the Indium Antimonide group, the Zinc Selenide Group,
 the Liquid Crystal Group, and the Holographic group. 

During the last stages of the Ph.D.  I was  working in the Statistical
Mechanics Group of the Mathematics Department. My work here was  writing computer
code to simulate spin exchange dynamics on lattices.
I returned to the physics
department to work on numerical solutions of Maxwell's equations
coupled to nonlinear diffusion equations in three dimensions. 

I came to Reading in 1992 to research into remote sensing. My main research areas over this
period were remote sensing instruments, satellite orbit mechanics, atmospheric scattering,
 surface reflectance and spectroscopy.
During this period I enjoyed a collaboration with Imperial College, and I had 3 Ph.D. students. 
Another smaller collaboration
was with  the University of Venice on the remote sensing of the Venice lagoon. 

Unfortunately, From 1997 to 2002, I was becoming increasingly ill with IBD, and
left ESSC in Sept 2002. 

\section{Computing Experience}

My computing experience is mainly in FORTRAN 77, C and C++. but I 
have been learning about  Java, Perl, PHP, HTML, and CGI. My main
experience is in scientific computing, and my main word processor has been 
LateX, which is a standard for publishing in scientific journals. Unfortunately
this has left me with virtually no experience of commercial computing, such as the standard 
Microsoft Word and Excel. However, I do not see my learning more commercial applications
as any kind of problem.

\section{Presentation Skills}

From my Ph.D. work onwards, I have published in the peer reviewed literature,
and  made poster and oral presentations
at conferences.  I have also given regular seminars at both Heriot-Watt University
and ESSC. This included all doing my own  graphical
and typographical work.  I have also made  contributions to 
grant proposals and annual reports. At ESSC,  I often made presentations to visiting guests from industry,
 NERC, and the House of Lords Science and Technology committee. As a visitor, I represented
 ESSC at Matra-Marconi, Imperial College, UCL, RAL, the Meteorological Research Flight, and the British Antarctic Survey.

\end{flushleft}

\begin{flushleft}
\section{References}
I shall be quite happy to supply references when required.
\end{flushleft}



\end{document}





